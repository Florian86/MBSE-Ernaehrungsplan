\documentclass[10pt, a4paper]{article}
\usepackage[a4paper, bottom=2.0cm, top=2.0cm]{geometry}

\usepackage[utf8]{inputenc}
\usepackage[ngerman]{babel}

\pagestyle{empty}

\parindent0pt

\usepackage{tabularx}
\usepackage{multirow}
\newcolumntype{C}{>{\centering\arraybackslash}X}

\usepackage{pdflscape}

\usepackage{ragged2e}

\usepackage{enumitem} 
\setitemize{leftmargin=*}
\begin{document}
\begin{landscape}

	{\Large \textbf{Ernährungsplan}} \medskip \\
	Max Mustermann \\
	Empfohlener Energiebedarf pro Woche: 14.000 Kalorien $\rightarrow$ 2.000 Kalorien pro Tag \medskip \\
	%Da in diesem Ernährungsplan nur Mittagessen betrachtet werden, für die jeweils 1.000 kcal veranschlagt werden, stehen 2.000-1.000 = 1.000 kcal zur freien Verfügung. \medskip \\
	\renewcommand*{\arraystretch}{1.2}
	\begin{tabularx}{\linewidth}{|X|X|X|X|X|X|X|}	
		\hline
		\Centering \multirow{2}{*}{\textbf{Montag}} & \Centering \multirow{2}{*}{\textbf{Dienstag}} & \Centering \multirow{2}{*}{\textbf{Mittwoch}} & \Centering \multirow{2}{*}{\textbf{Donnerstag}} & \Centering \multirow{2}{*}{\textbf{Freitag}} & \Centering \multirow{2}{*}{\textbf{Samstag}} & \Centering \multirow{2}{*}{\textbf{Sonntag}} \\
%		&  &  &  &  &  &  \\
		&  &  &  &  &  &  \\
		\hline
		Spaghetti Bolognese \newline {\scriptsize 1000 kcal} 
		\begin{small}
		\begin{itemize}
		\itemsep0pt
			\item 100g Spaghetti
			\item 200g Bolognesesoße \smallskip
		\end{itemize}
		\end{small}
		\begin{scriptsize}
		Anmerkung: Gericht ist in Buch blablabla auf S.30 zu finden.
		\end{scriptsize}
		& Gericht A \newline {\scriptsize 1000 kcal} 
		\begin{small}
		\begin{itemize}
		\itemsep0pt
			\item 500g Kartoffeln
			\item 200g Rührei 
			\item 100g Spinat
		\end{itemize}
		\end{small}
		& Pfannkuchen \newline {\scriptsize 500 kcal}  
		\begin{small}
		\begin{itemize}
		\itemsep0pt
			\item 300g Pfannkuchen
		\end{itemize}
		\end{small}
		&  Gericht A \newline {\scriptsize 1000 kcal}  
		\begin{small}
		\begin{itemize}
		\itemsep0pt
			\item 500g Kartoffeln
			\item 200g Rührei 
			\item 100g Spinat
		\end{itemize}
		\end{small}
		&  Gericht A \newline {\scriptsize 1000 kcal}  
		\begin{small}
		\begin{itemize}
		\itemsep0pt
			\item 500g Kartoffelbrei
			\item 200g Rührei 
			\item 100g Spinat
		\end{itemize}
		\end{small}
		&  Gericht A \newline {\scriptsize 1000 kcal}  
		\begin{small}
		\begin{itemize}
		\itemsep0pt
			\item 500g Zungenragout
			\item 200g Rührei 
			\item 100g Spinat
		\end{itemize}
		\end{small}
		&  Gericht A \newline {\scriptsize 1000 kcal}  
		\begin{small}
		\begin{itemize}
		\itemsep0pt
			\item 500g Kartoffeln
			\item 200g Rührei 
			\item 100g Rotkohl 
		\end{itemize}
		\end{small} 
		\\
		\hline
		& & Salat \newline {\scriptsize 300 kcal} 
		\begin{small}
		\begin{itemize}
		\itemsep0pt
			\item 500g Salat
			\item 50ml Dressing
		\end{itemize}
		\end{small}
		& & & & Salat \newline {\scriptsize 300 kcal}  
		\begin{small}
		\begin{itemize}
		\itemsep0pt
			\item 500g Salat
			\item 50ml Dressing
		\end{itemize}
		\end{small}
		\\
		\hline
	\end{tabularx} \medskip \\ 
Für diese Woche wurden xxxxx von 14.000 kcal verbraucht. 

\end{landscape}
\newpage
{\Large \textbf{Einkaufsliste}} \medskip \\

\begin{itemize}
	\item 100g Spaghetti ($\rightarrow$ 100 = Mo + Di + Mi + ... + So)
	\item 500g Kartoffeln 
	\item 
\end{itemize}
\end{document}
